\clearpage

\section{Simulation Analysis}
\label{sec:simulation}

The circuit, as is presented on Figure \ref{fig:circuit}, was simulated using \textit{NGSpice}.

\subsection{Voltage Gain}

We first took a look at what happened in node \textit{out}, which suffered the effect from both of the stages of the circuit, using frequency analysis. The result is presented in Figure \ref{fig:vo2f}.

\begin{figure}[h] \centering
\includegraphics[width=0.6\linewidth]{../sim/vo2f.pdf}
\caption{Output Voltage Gain Simulated}
\label{fig:vo2f}
\end{figure}

This graph is the consequence of a passband filter, which means that there is only a specific set of values - a band, if you will - where the circuit works the way it is supposed to. In this case, the passband filter works between around $10^4$ and $10^6$ Hz, which means the amplifier barely works for the human ear.

Additionaly, we worked out the bandwidth, which is shown in Table \ref{tab:bandwidth}.

\begin{tabular}{|l|r|}
  \label{tab:bandwidth}
  \hline    
  {\bf Variable} & {\bf Value} \\ \hline
  \input{../sim/bandwidth_tab.tex}
\end{tabular}

The value of the bandwidth shows for how long the passband filter works - i.e. for how much of the frequencies tested the circuit actually amplifies.

\subsection{Input and Output Impedances}

To work out the input impedance, we simply measured the voltage in node \textit{in2}, the node of the input stage of the circuit, and divided it by teh current going through it. The result is shown in Table \ref{tab:zin}.

\begin{tabular}{|l|r|}
\label{tab:zin}
  \hline    
  {\bf Variable} & {\bf Value} \\ \hline
  zin & 9.999982e+02,-3.99820e+02\\ \hline

\end{tabular}

For measuring the output impedance, we had to change the circuit used in \textit{NGSpice}.