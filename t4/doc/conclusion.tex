\clearpage
\section{Conclusion}
\label{sec:conclusion}

In conclusion, it was verified that, while the theoretical predictions are good to have an understanding of how the components interact with each other, it only works under specific conditions - in this case for lower frequencies - whereas the simulation always gives a better approximation of reality. This happens because the theoretical analysis is an oversimplification of the real circuit, in this case, switching the transistors for all linear components, in such a way that the circuit still, ultimately works as a filter. However, as we could see in the graph for the gain in the theoretical analysis, the circuit used for the theoretical analysis works more closely as a highpass filter instead of a bandpass one.

As such, we do believe that the analysis of this amplifier circuit was successful, even considering the differences between the two analysis.

\begin{thebibliography}{}

\bibitem{slides-ngspice}
Phyllis R. Nelson, \emph{Introduction to SPICE Source Files} Slides

\bibitem{spice-stanford}
\emph{SPICE 'Quick' Reference Sheet}, Stanford University

\bibitem{ngspice-guide}
Holger Vogt \textit{et al}, \emph{Ngspice's User Manual}, Version 34

\bibitem{octave}
\emph{GNU Octave} Documentation Files 

\bibitem{slides-prof}
José Teixeira de Sousa, \emph{Circuit Theory and and Eletronic Fundamentals} Class Slides

\end{thebibliography}