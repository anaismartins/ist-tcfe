\section{Simulation Analysis}
\label{sec:simulation}

\subsection{Operating Point Analysis}

Table~\ref{tab:op} shows the simulated operating point results for the circuit under analysis.

\begin{table}[htb!]
  \centering
  \begin{tabular}{|l|r|}
    \hline    
    {\bf Name} & {\bf Value [A or V]} \\ \hline
    @gb[i] & -2.48283e-04\\ \hline
@id[current] & 1.031476e-03\\ \hline
@r1[i] & 2.369026e-04\\ \hline
@r2[i] & -2.48283e-04\\ \hline
@r3[i] & -1.13802e-05\\ \hline
@r4[i] & 1.210639e-03\\ \hline
@r5[i] & 1.279759e-03\\ \hline
@r6[i] & 9.737369e-04\\ \hline
@r7[i] & 9.737369e-04\\ \hline
v(1) & 8.139731e+00\\ \hline
v(2) & 7.896243e+00\\ \hline
v(3) & 7.380521e+00\\ \hline
v(4) & 2.954689e+00\\ \hline
v(5) & 7.931699e+00\\ \hline
v(6) & 1.180782e+01\\ \hline
v(7) & 9.781670e-01\\ \hline
v(8) & 2.954689e+00\\ \hline

  \end{tabular}
  \caption{Operating point results. A variable preceded by @ is of type {\em current}
    and expressed in Ampere; other variables are of type {\it voltage} and expressed in
    Volt.}
  \label{tab:op}
\end{table}

These results were produced using the \textit{Ngspice software}. In order for the \textit{software} to be able to recognise the Current-Controlled Voltage Source, defined in Figure \ref{fig2} as $H_c$, we had to add a new Independent Voltage Source with a voltage of 0V, which is also represented in \ref{fig2}.


\begin{figure}[h] \centering
\includegraphics[width=0.4\linewidth]{t1-2.pdf}
\caption{The original circuit with an added voltage source of value 0V.}
\label{fig2}
\end{figure}

De um modo geral, os valores obtidos na análise teórica apresentam algumas diferenças comparativamente aos resultados da simulação.

No método dos nodos, em concreto, notam-se descrepancias muito acentuadas, sendo alguns dos casos de ordens de grandeza diferentes, tal como o valor de tensão do nodo 5: valor teórico - 0.0000000e+00 V, valor simulado - 7.931699e+00 V. Ainda assim, alguns valores aproximam-se moderadamente dos valores simulados - tensao no nodo 1: valor teórico -  5.1850419e+00 V, valor simulado - 8.139731e+00 V.



O método das malhas destaca-se significativamente do método anterior, já que apresenta resultados consideravelmente próximos dos valores simulados.
Os valores arbitrários de corrente atribuídos na figura \ref{fig2} (representados pelas setas circulares): $I_1$ $I_2$ $I_3$ e $I_4$ correspondem, respetivamente aos seguintes valores da tabela (\ref{tab:op})dos dados simulados: @r1[i], -@r2[i], -@id[current], -@r6[i] (ou @r7[i], são equivalentes).
Na verdade, o maior desvio absoluto corresponde à corrente $I_4$ e toma o valor: $ 5.7739e-05 $ A.