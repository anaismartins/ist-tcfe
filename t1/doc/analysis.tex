\section{Theoretical Analysis}
\label{sec:analysis}

In this section, the circuit shown in Figure~\ref{fig1} is analysed via node analysis and mesh analysis.

\subsection{Node analysis}

We have decided to numerate every node from 1 to 7 (the 8th of which being the ground). This is helpful because we can, therefore, derive direct equations in terms of the voltage at these nodes.
The voltage $V_i$ refers to the node $i$. Since we can only apply KCL (Kirchoff's Current Law), which states that the sum of currents leaving a node must be the same as the sum of currents entering a node, to nodes not connected to voltage sources, we can only derive 4 of these equations. The others are obtained from simple analysis of nodes connected to sources. The equations are as follows

\begin{equation} 
\begin{cases}  
    Node\, 2: \frac{V_1 - V_2}{R_1} + \frac{V_3 - V_2}{R_2} + \frac{V_5 - V_2}{R_3} = 0 \\
    Node\, 3: \frac{V_2 - V_3}{R_2} + K_bV_b = 0 \\
    Node\, 6: \frac{V_5 - V_6}{R_5} - K_bV_b = \,  - I_d \\
    Node\, 7: \frac{V_4 - V_7}{R_6} - \frac{V_7}{R_7} = 0 \\
    Caused\, by\, V_a : V_1 - V_4 \, = V_a \\
    Caused\, by\, V_b : \frac{V_3 - V_2}{R_2} - K_bV_b  = 0\\
    Caused\, by\, V_b : V_2 - V_5 - V_b = 0 \\
    Caused\, by\, V_c : V_5 - V_c = 0 \\
    Caused\, by\, I_c : - \frac{V_4 - V_7}{R_6} + \frac{V_c}{K_c} = 0\\
    
\end{cases}
\label{eq:1}
\end{equation}

We then use $GNU Octave$, a $software$ that can solve them, to obtain the values of all the unknowns. Knowing all the voltages allows us to know every current aswell, which means the circuit is solved.
The results of these computations are compiled in this table.

\begin{table}[h]
  \centering
  \begin{tabular}{|l|r|}
    \hline    
    {\bf Name} & {\bfValue [A and V]} \\ \hline
    V1 & 5.185041907790 A\\
V2 & -0.883725942007 A\\
V3 & -13.737754291516 A\\
V4 & 0.000000000000 A\\
V5 & 0.000000000000 A\\
V6 & 21.867128164958 A\\
V7 & 0.000000000000 A\\
Vb & -0.883725942007 A\\
Vc & 0.000000000000 A\\

  \end{tabular}
  \caption{Node analysis computation results. A variable starting with I is of type {\em current}
    and expressed in Ampere; a variable starting with V is of type {\it voltage} and expressed in
    Volt.}
  \label{tab:op}
\end{table}

\subsection{Mesh analysis}

We have decided to numerate every mesh from 1 to 4. This is helpful because we can, therefore, derive direct equations in terms of the current passing through these meshes. The current $I_i$ refers to the mesh $i$. For this method, we apply KVL (Kirchoff's Voltage Law), which states that the sum of all the voltages around any closed loop in a circuit is equal to zero, and relate the fictional currents we created to currents given in the circuit (for example, $I_d$). The equations are as follows

\begin{equation} 
\begin{cases}  
    Mesh\, 1: R_1\,I_1 + V_b + R_4(I_1 - I_3) = V_a \\
    Mesh\, 2: I_2 = -I_b = -K_b\, V_b\\
    Mesh\, 3: R_4(I_3 - I_1) + K_c\,I_c + R_7\,I_3 + R_6\,I_3 = 0 \\
    Mesh\, 4: I_4 = -I_d \\
    I_3 = -I_c \\
    V_b = R_3(I_1 - I_2) \\
    
    
\end{cases}
\label{eq:2}
\end{equation}

We use the same procedure to compute these equations. The results of these computations are compiled in this table.

\begin{table}[h]
  \centering
  \begin{tabular}{|l|r|}
    \hline    
    {\bf Name} & {\bfValue [A or V]} \\ \hline
    I1 & 0.000236902599 A\\
I2 & 0.000248282786 A\\
I3 & -0.000973736898 A\\
I4 & -0.001031475905 A\\
Ib & -0.000248282786 A\\
Ic & 0.000973736898 A\\

  \end{tabular}
  \caption{Mesh analysis computation results. A variable starting with I is of type {\em current}
    and expressed in Ampere; a variable starting with V is of type {\it voltage} and expressed in
    Volt.}
  \label{tab:op}
\end{table}
