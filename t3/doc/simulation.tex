\clearpage

\section{Simulation Analysis}
\label{sec:simulation}

In Figure \ref{fig:circuit-ngspice}, the circuit used by \textit{Ngspice} is displayed. Instead of the transformer, we use one voltage controlled voltage source and one current controlled current source to simulate an ideal transformer. Furthermore, we add a 0V voltage source between nodes 2 and 4 because \textit{Ngspice} needs it for the current source to work. Moreover, the diodes used are the default ones created by the \textit{software}.

\begin{figure}[h] \centering
\includegraphics[width=0.6\linewidth]{circuit-ngspice.pdf}
\caption{The circuit we will be working with in Ngspice.}
\label{fig:circuit-ngspice}
\end{figure}

After playing around with the numbers for a while to make sure the output voltage was 12V and approximately DC, we ended up with the following values for our circuit:

\begin{tabular}{|l|r|}
  \hline    
  {\bf Variable} & {\bf Value} \\ \hline
  \input{../sim/out_TAB}
\end{tabular}

\begin{tabular}{|l|r|}
  \hline    
  {\bf Variable} & {\bf Value} \\ \hline
  \input{../sim/ripple.tex}
\end{tabular}

\begin{tabular}{|l|r|}
  \hline    
  {\bf Variable} & {\bf Value} \\ \hline
  \input{../sim/merit.tex}
\end{tabular}

Considering the goal was to get an average of 12V, this result is extremely satisfactory. The ripple is also very low, which, once again, is extremely satisfactory. We then conclude that the simulation analysis was successful.

To add to this, we got the following plots for the envelope detector and voltage regulator circuits (Figures \ref{fig:sim-envelope} and \ref{fig:sim-regulator} respectively).

\begin{figure}[h] \centering
\includegraphics[width=0.6\linewidth]{../sim/sim-envelope.pdf}
\caption{Envelope detector output from the simulation analysis.}
\label{fig:sim-envelope}
\end{figure}

\begin{figure}[h] \centering
\includegraphics[width=0.6\linewidth]{../sim/sim-regulator.pdf}
\caption{Voltage regulator output from the simulation analysis.}
\label{fig:sim-regulator}
\end{figure}