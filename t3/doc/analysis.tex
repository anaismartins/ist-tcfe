\section{Theoretical Analysis}
\label{sec:analysis}

\subsection{Envelope Detector}

The envelope detector circuit takes advantage of the capacitor properties to flaten the AC signal recieved from the transformer, approximating it to a DC current. This circuit is periodic in time and has two very distinct phases, associated to the diodes characteristics: $D_{on}$ - meaning the diode is active and, therefore, behaves like a shunt and $D_{off}$ - the diode opens the circuit.  To analyse the behaviour of this portion of the AC/DC converter we must study the envelope detector voltage, i.e the voltage at the capacitors terminals, $v_0$. According to the description above we get:

\begin{center}
$
v_0 = 
\begin{cases} 
v_s \, , \, D_{on} \\ - R i_c \, , \, D_{off}\\ 
\end{cases}
$
\end{center}

, where $i_c$, the current in the capacitor's branch is given by: $i_c = C\frac{dv_0}{dt}$ and $v_s$ is the voltage recieved from the transformer: $v_s(t) = V_scos(\omega t)$.

\subsubsection{$D_{ON}$}

At $D_{on}$, as seen, $v_0(t) = v_s(t)$ , $i_c = C\frac{dv_s}{dt}$ and, by Ohm's law, $i_R = \frac{v_s}{R}$.\par

Further analysing the capacitors current, we obtain:


\begin{center}
	\begin{equation}
	i_c = -C.V_s.\omega sin(\omega t)
	\end{equation}
\end{center}

Which grows negative with the time derivative. This means that, periodically, $i_c$ contributes to the accumumlation of voltage in the $5^{th}$ node, which, at some point, turns the rectifier bridge off: at $t_{off}$, where the voltage in node 5 is greater than the voltage in node 3, and the diode in between opens the circuit.


*Note that, in $D_{ON}$, as the diode acts like a shunt: $i_D = i_C + i_R$.

\subsubsection{$D_{OFF}$}

Having the diode between nodes 3 and 5 turned on means that the currente passing in its branch is null ($i_D = 0 $) so: $i_R = -i_c$. This expression can be used to calculate $t_{off}$ theoretically. In the limit we have $v_0 = v_s$ which, applied to the previous expression, yields:


\begin{center}
\begin{equation}
	\frac{V_{s}}{R} cos(\omega t_{off}) = C.V_s.\omega sin(\omega t_{off}) \Rightarrow t_{off} = \frac{1}{\omega} atg(\frac{1}{\omega RC})
\end{equation}
\end{center}

----------------------------------VALOR OBTIDO NO OCTAVE

From this moment on, with the circuit cut off at node 5, the capacitor unloads the accumulated voltage through the resistor, until the diode turns itself on again.\par

Solving the differential equation ($v_0(t) = C\frac{dv_0(t)}{dt}$), in t, $t_{off} < t < t_{on}$, we get:

\begin{center}
\begin{equation}
	v_0(t) = V_s cos(\omega t_{off}).e^{-\frac{ t-t_{off} }{RC}}
\end{equation}
\end{center}


$t_{on}$ is hit when the voltage in 5 evens and gets smaller than voltage 3 again, i.e the voltage in the envelope detector circuit's input equals its output. To compute this instant we must apply:

\begin{center}
\begin{equation}
	 V_s cos(\omega t_{on}) = V_s cos(\omega t_{on}).e^{-\frac{ t_{on}-t_{off} }{RC}}
\end{equation}
\end{center}

--------------------------------------VALOR OBTIDO NO OCTAVE

\subsubsection{Envelope Detector Plot}

In other words, the ploted envelope detector voltage should be, in $t_{on} < t < t_{off}$, a sinusoidal wave form (cossine), whereas, in $t_{off} < t < t_{on}$, a decaying exponencial, which approximates the whole sinusoidal wave (AC current)to a straight line (DC current).


---------------------------------PLOT DO ENVELOPE OBTIDO NO OCTAVE

\subsection{Voltage Regulator}



