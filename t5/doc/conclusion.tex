\clearpage
\section{Conclusion}
\label{sec:conclusion}

In conclusion, it was verified that the theoretical analysis and simulation were very similar, leading to the assessment that the OP-AMP is a considerably linear component. Furthermore, we also concluded that these approximations are much more viable for lower frequencies than for higher frequencies (starting at around 10$^4$Hz is where you can see the most difference).
Even thought the gain was not achieved in the desired central frequency, we were still able to come very close to the central frequency, which was just a few Hz after our upper cutoff frequency.

In general, the study of this circuit, which focused mainly on the usages of an OP-AMP, was a success.

\begin{thebibliography}{}

\bibitem{slides-ngspice}
Phyllis R. Nelson, \emph{Introduction to SPICE Source Files} Slides

\bibitem{spice-stanford}
\emph{SPICE 'Quick' Reference Sheet}, Stanford University

\bibitem{ngspice-guide}
Holger Vogt \textit{et al}, \emph{Ngspice's User Manual}, Version 34

\bibitem{octave}
\emph{GNU Octave} Documentation Files 

\bibitem{slides-prof}
José Teixeira de Sousa, \emph{Circuit Theory and and Eletronic Fundamentals} Class Slides

\bibitem{guide}
Panquecas:
\par
225g unsalted butter, softened 310g caster sugar 4 eggs 225g self-raising flour 2.5 lemons :lemon: 

\end{thebibliography}

